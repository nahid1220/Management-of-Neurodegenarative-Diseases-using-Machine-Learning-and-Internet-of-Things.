Neurodegenarative Disease(NDD) deemed to be one of the critical problems for the elderly patient as there are currently no successful treatments, early diagnosis and the avoidance of wrong diagnosis tend to be crucial to maintaining a decent quality of life for patients. It turns out to be a problem for human health and draws researchers to control NDD via wearable, compact, and imaging sensing instruments. With the emergence of low cost widespread sensing components, the development of ubiquitous computing and a deeper understanding of machine learning methods, researchers have used numerous machine learning approaches in managing NDD from the sensor data. In this paper, We have proposed a deep learning framework to handle a patient's everyday life with a neurological disorder using the Internet of Things and then manage the patient by referring to the doctor. The deep learning based model fused knowledge from both the smartphone/wearable and camera installed on the wall and ceiling. Audio signal is also integrated with the model to perform better performance in the management of NDD. A Recurrent Neural Network based fall detection method is also introduced in our model.









\vspace{8pt}
\textbf{Keywords:} Sensor, Wearable device, IoT, Assistive Living 



